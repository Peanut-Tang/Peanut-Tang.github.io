\documentclass[a4paper,12pt]{ctexart}

% 行间距调整
\usepackage{setspace}
\setstretch{1.1}

%----------------------------------------------------------------------------------------
%	FONT
%----------------------------------------------------------------------------------------

% font options
\usepackage{newpxtext}
% \linespread{1.05}  % palladio needs more leading (space between lines)
\usepackage[T1]{fontenc}

% % fontspec allows you to use TTF/OTF fonts directly
% \usepackage{fontspec}
% \defaultfontfeatures{Ligatures=TeX}

% % modified for ShareLaTeX use
% \setmainfont[
% SmallCapsFont = Fontin-SmallCaps.otf,
% BoldFont = Fontin-Bold.otf,
% ItalicFont = Fontin-Italic.otf
% ]
% {Fontin.otf}

%----------------------------------------------------------------------------------------
%	PACKAGES
%----------------------------------------------------------------------------------------
\usepackage{url}
\usepackage{parskip} 	

%other packages for formatting
\RequirePackage{color}
\RequirePackage{graphicx}
\usepackage[usenames,dvipsnames]{xcolor}
\usepackage[scale=0.9]{geometry}

%tabularx environment
\usepackage{tabularx}

%for lists within experience section
\usepackage{enumitem}

% centered version of 'X' col. type
\newcolumntype{C}{>{\centering\arraybackslash}X} 

%to prevent spillover of tabular into next pages
\usepackage{supertabular}
\usepackage{tabularx}
\newlength{\fullcollw}
\setlength{\fullcollw}{0.47\textwidth}

%custom \section
\usepackage{titlesec}				
\usepackage{multicol}
\usepackage{multirow}

%CV Sections inspired by: 
%http://stefano.italians.nl/archives/26
\titleformat{\section}{\Large\scshape\raggedright}{}{0em}{}[\titlerule]
\titlespacing{\section}{0pt}{10pt}{10pt}

%for publications
\usepackage[style=authoryear,sorting=ynt, maxbibnames=2]{biblatex}

%Setup hyperref package, and colours for links
\usepackage[unicode, draft=false]{hyperref}
\definecolor{linkcolour}{rgb}{0,0.2,0.6}
\hypersetup{colorlinks,breaklinks,urlcolor=linkcolour,linkcolor=linkcolour}
\addbibresource{citations.bib}
\setlength\bibitemsep{1em}

%for social icons
\usepackage{fontawesome5}

%debug page outer frames
%\usepackage{showframe}

%----------------------------------------------------------------------------------------
%	BEGIN DOCUMENT
%----------------------------------------------------------------------------------------
\begin{document}

% non-numbered pages
\pagestyle{empty} 

%----------------------------------------------------------------------------------------
%	TITLE
%----------------------------------------------------------------------------------------

% \begin{tabularx}{\linewidth}{ @{}X X@{} }
% \huge{Your Name}\vspace{2pt} & \hfill \emoji{incoming-envelope} email@email.com \\
% \raisebox{-0.05\height}\faGithub\ username \ | \
% \raisebox{-0.00\height}\faLinkedin\ username \ | \ \raisebox{-0.05\height}\faGlobe \ mysite.com  & \hfill \emoji{calling} number
% \end{tabularx}

\begin{tabularx}{\linewidth}{@{} C @{}}
\Huge{\scshape Shengtang Huang (黄盛唐)} \\[7.5pt]
\raisebox{-0.05\height}\faPhone \ (+86) 188-0500-6918 \quad \faHome \ 96 Jinzhai Road, Hefei, 230026, China \\
\raisebox{-0.05\height}\faEnvelope \ \href{mailto:peanuttang@mail.ustc.edu.cn}{peanuttang@mail.ustc.edu.cn}, \href{mailto:peanuttang1320061044@gmail.com}{peanuttang1320061044@gmail.com} \\
% \raisebox{-0.05\height}\faGlobe \ Link: \href{https://shengtanghuang.netlify.app/about/}{Personal Website}
\end{tabularx}



%----------------------------------------------------------------------------------------
% EXPERIENCE SECTIONS
%----------------------------------------------------------------------------------------

% %Interests/ Keywords/ Summary
% \section{\textbf{Research Interests}}
% \textbf{Theoretical Computer Science:} pseudorandomness, computational complexity.

% \textbf{Mathematics:} combinatorics and graph theory, probability theory.


%----------------------------------------------------------------------------------------
%	EDUCATION
%----------------------------------------------------------------------------------------
\section{\textbf{Education}}
\textbf{University of Science and Technology of China (USTC)}, Hefei, China \hfill \textit{Aug. 2022 $\sim$ Present}  \vspace{-0.1cm}
\begin{itemize}
    \item Bachelor of Computer Science and Technology \hfill \textit{currently a senior student}  \vspace{-0.25cm}
    \item School of the Gifted Young \hfill \textit{supervised by Prof. Xue Chen}  \vspace{-0.25cm}
    \item \textbf{Overall GPA: 3.88/4.30, top 5\%} \hfill \textbf{Major GPA: 4.01/4.30}  \vspace{-0.25cm}
    \item \textbf{Core Courses:}  \vspace{-0.25cm}
    
          \textbf{Computer Science:} Foundations of Algorithms (100), Computer Programming (H) (99), Data Structures (97), Computer Organization (95), Computer Networks (94).  \vspace{-0.25cm}

          \textbf{Mathematics:} Linear Algebra I \& II (90, 92), Advanced Combinatorics (100), Probability Theory \& Its Outer Chapter (100, 100), Advanced Probability Theory *(100), Graph Theory *(92), Regression Analysis (97), Operations Research (96), Optimization Algorithm *(92).  \vspace{-0.25cm}

          \textbf{Note:} (H) represents the curriculum of Honors. * indicates that this course is a graduate-level course.
\end{itemize}



\section{\textbf{Publications \& Manuscripts}}

\textbf{Following the convention in theoretical computer science, unless stated otherwise, author names are ordered \underline{alphabetically}.}  \vspace{-0.1cm}

\begin{enumerate}
    \item \textbf{Range Avoidance and Remote Point: New Algorithms and Hardness} \\
           \underline{Shengtang Huang}, Xin Li, Yan Zhong \\
           In the 17th Innovations in Theoretical Computer Science Conference, \textbf{ITCS 2026}

    \item \textbf{Explicit Min-wise Hash Families with Optimal Size} \\
           Xue Chen, \underline{Shengtang Huang}, Xin Li \\
           In the 37th Annual ACM-SIAM Symposium on Discrete Algorithms, \textbf{SODA 2026}
\end{enumerate}



%Experience
\section{\textbf{Research Experiences}}

\textbf{Derandomization of load balancing based on linear hashing} \\
\textit{Advisors: Xue Chen, USTC; Xin Li, JHU; Fernando Granha Jeronimo, UIUC \hfill Jul. 2025 $\sim$ Present}  \vspace{-0.1cm}
\begin{itemize}
    \item Tried to find an explicit linear hash family with small size that has nice load balancing properties.
\end{itemize}

\textbf{Construction of pseudorandom code against insertion-deletion error} \\
\textit{Advisor: Xin Li, JHU \hfill Oct. 2025 $\sim$ Present}  \vspace{-0.1cm}
\begin{itemize}
    \item Tried to construct a pseudorandom code (PRC) against insertion-deletion error. Previous results only construct PRCs against substitution error.
\end{itemize}

\textbf{New algorithms and hardness for range avoidance and remote point problems} \\
\textit{Advisor: Xin Li, JHU \hfill Jul. 2025 $\sim$ Sept. 2025}  \vspace{-0.1cm}
\begin{itemize}
    \item Proved the equivalence between the existence of $\textbf{FP}^{\textbf{NP}}$ algorithms for $\mathcal{C}$-$\textsc{Avoid}[n, n^{1 + \varepsilon}]$ problem and $2^{\Omega(n)}$ lower bounds against $\mathcal{C}$ circuits for the class $\textbf{E}^{\textbf{NP}}$, for some suitable circuit class $\mathcal{C}$.
    \item Showed the equivalence between the existence of $\textbf{FP}^{\textbf{NP}}$ algorithms for general $\textsc{RemotePoint}$ problem and $2^{\Omega(n)}$ average-case lower bounds against general circuits for the class $\textbf{E}^{\textbf{NP}}$.
    \item Designed a fast graph-based algorithm for $\textbf{NC}^0_k$-$\textsc{Avoid}[n, n^{1 + \varepsilon}]$, with time complexity $2^{n^{1 - \frac{\varepsilon}{k - 1} + o(1)}}$.
    \item Found a greedy algorithm for $\textbf{NC}^0_k$-$\textsc{Avoid}[n, n + 1]$, with time complexity $O(n 2^{\frac{k - 1}{k - 2} n})$.
    \item The paper has been accepted by \textbf{ITCS 2026}.
\end{itemize}

\textbf{Construction of the min-wise hash family with short seed length and small multiplicative error} \\
\textit{Advisors: Xue Chen, USTC; Xin Li, Johns Hopkins University \hfill Sept. 2024 $\sim$ Apr. 2025}  \vspace{-0.1cm}
\begin{itemize}
    \item Found the connection between the min-wise hash family and pseudorandomness generator of combinatorial rectangles.  \vspace{-0.25cm}
    \item Constructed an explicit min-wise hash family $\mathcal{H} = \{h : [N] \to [\mathrm{poly}(N)]\}$ with seed length $O(\log N)$ and multiplicative error $2^{-O\left(\frac{\log N}{\log \log N}\right)}$.  \vspace{-0.25cm}
    \item Designed an explicit $k$-min-wise hash family $\mathcal{H} = \{h : [N] \to [\mathrm{poly}(N)]\}$ with seed length $O(k \log N)$ and multiplicative error $2^{-O\left(\frac{\log N}{\log \log N}\right)}$.  \vspace{-0.25cm}
    \item The paper has been accepted by \textbf{SODA 2026}.
\end{itemize}

\textbf{A fast algorithm for $(1 + \varepsilon)$-approximate incremental matching problem on general graphs} \\
\textit{Advisor: Slobodan Mitrović, UCD \hfill Jul. 2024 $\sim$ Aug. 2024}  \vspace{-0.1cm}
\begin{itemize}
    \item Extended the method of solving $(1 + \varepsilon)$-approximate incremental matching problem from bipartite graphs to general graphs.  \vspace{-0.25cm}
    \item Achieved the amortized complexity $\exp(1 / \varepsilon)$ for $(1 + \varepsilon)$-approximate incremental matching problem.  \vspace{-0.25cm}
    \item Attended the workshop \href{https://simons.berkeley.edu/workshops/workshop-local-algorithms-wola}{WoLA} at Simons Institute for the Theory of Computing, University of California, Berkeley, with Prof. Slobodan Mitrović and Dr. Wen-Horng Sheu.
\end{itemize}




\section{\textbf{Teachings}}
\textbf{Teaching Assistant} \hfill \textit{USTC, Hefei, China}  \vspace{-0.1cm}
\begin{itemize}
    \item Operations Research (2024 Fall)  \hfill \textit{Lecturer: Prof. Shixiang Chen} \vspace{-0.25cm}
    \item Foundations of Algorithms (2024 Spring)  \makebox[10cm][r]{\textit{Lecturer: Prof. Xue Chen}}
\end{itemize}




\section{\textbf{Honors \& Awards}}
\begin{itemize}
    \item \textbf{Merit Award} in S.-T. Yau College Student Mathematics Contests 2025, \textbf{ranked 32nd} in Probability and Statistics track. The contest's difficulty is comparable to qualifying exams for Ph.D. programs at top U.S. universities. \vspace{-0.25cm}
    \item \textbf{Gold Prize for Outstanding Student Scholarship:} Awarded in Oct. 2025, this university-level scholarship is granted to the top 3\% of outstanding students at USTC. \vspace{-0.25cm}
    \item \textbf{Qiangwei Yuanzhi Scholarship:} Awarded in Oct. 2024, this university-level scholarship is granted to the top 5\% of outstanding students at USTC. \vspace{-0.25cm}
    \item \textbf{Sliver Awards} in the 2024 ICPC East Asia Shanghai Regional Contest and the 2023 ICPC East Asia Shenyang Regional Contest. \vspace{-0.25cm}
    \item \textbf{First Prize} in National Olympiad in Informatics in Provinces in 2020 and 2021.
\end{itemize}




% Skills Section
\section{\textbf{Skills}}
\begin{itemize}
    \item \textbf{Programming:} C, C++, Python, Matlab, R, Verilog. \vspace{-0.25cm}
    \item \textbf{Software:} Git, \LaTeX, Microsoft Office. \vspace{-0.25cm}
    \item \textbf{Languages:}  English (fluent), Chinese (native).
\end{itemize}


% %Projects
% \section{Projects}

% \begin{tabularx}{\linewidth}{ @{}l r@{} }
% \textbf{Some Project} & \hfill \href{https://some-link.com}{Link to Demo} \\[3.75pt]
% \multicolumn{2}{@{}X@{}}{long long line of blah blah that will wrap when the table fills the column width long long line of blah blah that will wrap when the table fills the column width long long line of blah blah that will wrap when the table fills the column width long long line of blah blah that will wrap when the table fills the column width}  \\
% \end{tabularx}


% \section{Selected Courses}
% \begin{tabularx}{1\textwidth}{
% >{\raggedright\arraybackslash}X 
% >{\raggedright\arraybackslash}X }
%     \begin{itemize}
%         \item Foundations of Algorithms (100)
%         \item Computer Programming (H) (99)
%         \item Data Structures (97)
%         \item Computer Organization (95)
%         \item Graph Theory* (92)
%         \item Regression Analysis (97)
%         \item Analysis of Boolean Functions (Audit, CMU)
%     \end{itemize}
%     & \begin{itemize}
%         \item Operations Research (96)
%         \item Optimization Algorithm* (92)
%         \item Probability Theory \& Its Outer Chapter (100, 100)
%         \item Advanced Probability Theory (100)
%         \item Graph Theory and Additive Combinatorics (Audit, MIT)
%     \end{itemize}
% \end{tabularx}

% \textbf{Note:} (H) represents the curriculum of Honors. * indicates that this course is a graduate-level course.




%----------------------------------------------------------------------------------------
%	PUBLICATIONS
%----------------------------------------------------------------------------------------
% \section{Publications}
% \begin{refsection}[citations.bib]
% \nocite{*}
% \printbibliography[heading=none]
% \end{refsection}

\section{\textbf{References}}

\begin{tabularx}{1.1\textwidth}{
    >{\raggedright\arraybackslash}X 
    >{\raggedright\arraybackslash}X }
        \textbf{Prof. Xue Chen (陈雪)}                                 & \textbf{Prof. Shixiang Chen (陈士祥)} \\
        Specially Appointed Professor                                  & Assistant Professor \\
        School of Computer Science and Technology                      & School of Mathematical Sciences \\
        University of Science and Technology of China                  & University of Science and Technology of China \\
        \href{mailto:xuechen1989@ustc.edu.cn}{xuechen1989@ustc.edu.cn} & \href{mailto:shxchen@ustc.edu.cn}{shxchen@ustc.edu.cn} \\
         & \\
        \textbf{Prof. Xin Li (李昕)}                                   & \textbf{Prof. Slobodan Mitrović} \\
        Associate Professor                                            & Assistant Professor \\
        Department of Computer Science                                 & Department of Computer Science \\
        Johns Hopkins University (JHU)                                 & University of California, Davis (UCD) \\
        \href{mailto:lixints@cs.jhu.edu}{lixints@cs.jhu.edu}           & \href{mailto:smitrovic@ucdavis.edu}{smitrovic@ucdavis.edu} \\
         & \\
        \textbf{Prof. Fernando Granha Jeronimo}                        &  \\
        Assistant Professor                                            &  \\
        Department of Computer Science                                 &  \\
        University of Illinois Urbana-Champaign (UIUC)                 &  \\
        \href{mailto:granha@illinois.edu}{granha@illinois.edu}         & 
\end{tabularx}

\end{document}